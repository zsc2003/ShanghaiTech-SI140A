\documentclass{article}

\usepackage{fancyhdr}
\usepackage{extramarks}
\usepackage{amsmath}
\usepackage{amsthm}
\usepackage{amsfonts}
\usepackage{tikz}
\usepackage[plain]{algorithm}
\usepackage{algpseudocode}
\usepackage{enumerate}
\usepackage{tikz}

\usetikzlibrary{automata,positioning}

%
% Basic Document Settings
%  

\topmargin=-0.45in
\evensidemargin=0in
\oddsidemargin=0in
\textwidth=6.5in
\textheight=9.0in
\headsep=0.25in

\linespread{1.1}

\pagestyle{fancy}
\lhead{\hmwkAuthorName}
\chead{\hmwkClass : \hmwkTitle}
\rhead{\firstxmark}
\lfoot{\lastxmark}
\cfoot{\thepage}

\renewcommand\headrulewidth{0.4pt}
\renewcommand\footrulewidth{0.4pt}

\setlength\parindent{0pt}

%
% Create Problem Sections
%

\newcommand{\enterProblemHeader}[1]{
    \nobreak\extramarks{}{Problem \arabic{#1} continued on next page\ldots}\nobreak{}
    \nobreak\extramarks{Problem \arabic{#1} (continued)}{Problem \arabic{#1} continued on next page\ldots}\nobreak{}
}

\newcommand{\exitProblemHeader}[1]{
    \nobreak\extramarks{Problem \arabic{#1} (continued)}{Problem \arabic{#1} continued on next page\ldots}\nobreak{}
    \stepcounter{#1}
    \nobreak\extramarks{Problem \arabic{#1}}{}\nobreak{}
}

\newcommand*\circled[1]{\tikz[baseline=(char.base)]{
		\node[shape=circle,draw,inner sep=2pt] (char) {#1};}}


\setcounter{secnumdepth}{0}
\newcounter{partCounter}
\newcounter{homeworkProblemCounter}
\setcounter{homeworkProblemCounter}{1}
\nobreak\extramarks{Problem \arabic{homeworkProblemCounter}}{}\nobreak{}

%
% Homework Problem Environment
%
% This environment takes an optional argument. When given, it will adjust the
% problem counter. This is useful for when the problems given for your
% assignment aren't sequential. See the last 3 problems of this template for an
% example.
%

\newenvironment{homeworkProblem}[1][-1]{
    \ifnum#1>0
        \setcounter{homeworkProblemCounter}{#1}
    \fi
    \section{Problem \arabic{homeworkProblemCounter}}
    \setcounter{partCounter}{1}
    \enterProblemHeader{homeworkProblemCounter}
}{
    \exitProblemHeader{homeworkProblemCounter}
}

%
% Homework Details
%   - Title
%   - Class
%   - Due date
%   - Name
%   - Student ID

\newcommand{\hmwkTitle}{Homework\ \#05}
\newcommand{\hmwkClass}{Probability \& Statistics for EECS}
\newcommand{\hmwkDueDate}{March 19, 2023}
\newcommand{\hmwkAuthorName}{Zhou Shouchen}
\newcommand{\hmwkAuthorID}{2021533042}


%
% Title Page
%

\title{
    \vspace{2in}
    \textmd{\textbf{\hmwkClass:\\  \hmwkTitle}}\\
    \normalsize\vspace{0.1in}\small{Due\ on\ \hmwkDueDate\ at 23:59}\\
	\vspace{4in}
}

\author{
	Name: \textbf{\hmwkAuthorName} \\
	Student ID: \hmwkAuthorID}
\date{}

\renewcommand{\part}[1]{\textbf{\large Part \Alph{partCounter}}\stepcounter{partCounter}\\}

%
% Various Helper Commands
%

% Useful for algorithms
\newcommand{\alg}[1]{\textsc{\bfseries \footnotesize #1}}
% For derivatives
\newcommand{\deriv}[1]{\frac{\mathrm{d}}{\mathrm{d}x} (#1)}
% For partial derivatives
\newcommand{\pderiv}[2]{\frac{\partial}{\partial #1} (#2)}
% Integral dx
\newcommand{\dx}{\mathrm{d}x}
% Alias for the Solution section header
\newcommand{\solution}{\textbf{\large Solution}}
% Probability commands: Expectation, Variance, Covariance, Bias
\newcommand{\E}{\mathrm{E}}
\newcommand{\Var}{\mathrm{Var}}
\newcommand{\Cov}{\mathrm{Cov}}
\newcommand{\Bias}{\mathrm{Bias}}

\begin{document}

\maketitle

\pagebreak

\begin{homeworkProblem}[1]
(a) Let $X$ be the number of questions until Kratos is sure about the location of the treasure.\\
So when $X = 1$, Kratos asks the correct location in one time, so $P(X = 1) = \dfrac{1}{9}$.\\
When $X = 2$, Kratos asks the wrong position at the first time, and asks the correct location in the second time, so $P(X = 2) = \dfrac{8}{9}\cdot \dfrac{1}{8} = \dfrac{1}{9}$.\\
Similarly, $P(X = 3) = P(X = 4) = P(X = 5) = P(X = 6) = P(X = 7) = \dfrac{1}{9}$.\\
As for $X = 8$, it is special since if Kratos wrongly asks the location for $8$ times, he can still get the correct location with using exclusion method, 
because there all totally $9$ realms. So $P(X = 8) = \dfrac{2}{9}$.\\

So $E(X)=\sum\limits_{x=1}^{8}xP(X=x) = \sum\limits_{x=1}^{7}x\cdot \dfrac{1}{9} + 8\cdot \dfrac{2}{9} = \dfrac{44}{9}$.\\
So above all, the expected number of questions until Kratos is sure about the location is $\dfrac{44}{9}$.\\
    
(b) Let $X$ be the number of questions until Kratos is sure about the location of the treasure.\\
Since Kratos elimainates as close to half of the remaining realms as possible, 
for each subsequence, we eliminate the $\lfloor \dfrac{length}{2} \rfloor$, where "length" is the length of the subsequence.\\
For example, firstly the subsequence is $[1,9]$, we ask if it is in a realm numbered less than or equal to $1 + \lfloor \dfrac{9}{2} \rfloor - 1 = 4$.\\
If so, the second time the subsequence is $[1,4]$, we will ask if it is in a realm numbered less than or equal to $1 + \lfloor \dfrac{4}{2} \rfloor - 1= 2$.\\
Otherwise, the second time the subsequence is $[5,9]$, we will ask if it is in a realm numbered less than or equal to $5 + \lfloor \dfrac{4}{2} \rfloor - 1= 6$.\\
And we will continously do this until the length of the subsequence is equal to $1$, then we will be sure that it is in that realm.\\
So to conclude, if it is in realm $1,2,3,4,5,6,9$, then we will ask $3$ times.\\
If it is in realm $7, 8$, then we will ask $4$ times.\\
So $P(X = 3) = \dfrac{7}{9}, P(X = 4) = \dfrac{2}{9}$.\\
So $E(X) = 3\cdot \dfrac{7}{9} + 4\cdot \dfrac{2}{9} = \dfrac{29}{9} $.\\
So above all, the expected number of questions until Kratos is sure about the location is $\dfrac{29}{9}$.\\


\end{homeworkProblem}

\newpage

\begin{homeworkProblem}[2]
Let $N_1$ : \# of days until first new type of steak appears.\\
Let $N_2$ : \# of days until the second new type of steak appears.\\
Let $N_3$ : \# of days until the third new type of steak appears.\\
Let $N_4$ : \# of days until the fourth new type of steak appears.\\
Let $N_5$ : \# of days until the fifth new type of steak appears.\\
So the number of days that we need to watch is $N = N_1 + N_2 + N_3 + N_4 + N_5$.\\
From the description of the problem, we can get that\\
$N_1 = 1, N_2\sim FS(\dfrac{4}{5}), N_3\sim FS(\dfrac{3}{5}), N_4\sim FS(\dfrac{2}{5}), N_5\sim FS(\dfrac{1}{5})$.\\
So $E(N_1) = 1, E(N_2) = \dfrac{5}{4}, E(N_3) = \dfrac{5}{3}, E(N_4) = \dfrac{5}{2}, E(N_5) = \dfrac{5}{1}$.\\
So $E(N) = E(N_1 + N_2 + N_3 + N_4 + N_5) = E(N_1) + E(N_2) + E(N_3) + E(N_4) + E(N_5) = 1 + \dfrac{5}{4} + \dfrac{5}{3} + \dfrac{5}{2} + \dfrac{5}{1} = \dfrac{137}{12}$.\\

So above all, the expectation of the days that we expect to watch is $\dfrac{137}{12}$.\\

\end{homeworkProblem}

\newpage

\begin{homeworkProblem}[3]
(a) Let $Z$ be the number of that the first time thay simultaneously successful.\\
Since $X_i\sim Bern(p_1), Y_i\sim Bern(p_2)$,\\
so for one time, they have the possibility of $p_1p_2$ to success.\\
So $Z\sim FS(p_1p_2)$, and $P(Z=k) = (1-p_1p_2)^{k-1}(p_1p_2)$, and the support is $\{1,2,\cdots\}$.\\
And from what we have learned, the expectation of $Z\sim FS(p_1p_2)$ is $E(Z) = \dfrac{1}{p_1p_2}$.\\
So above all, the distribution of $Z$ is $Z\sim FS(p_1p_2)$, and the expectation is $E(Z)=\dfrac{1}{p_1p_2}$.\\

(b) Let $A$ be the first time that at least one has a success.\\
And the probability that at least one success is that $1-(1-p_1)(1-p_2) = p_1 + p_2 - p_1p_2$.\\
So we can say that $A\sim FS(p_1+p_2-p_1p_2)$.\\
And $P(A=k)=(1-p_1-p_2+p_1p_2)^{k-1}(p_1+p_2-p_1p_2)$.\\
And from what we have learned in class, we could get that the expectation of $A$ is that 
$E(A) = \dfrac{1}{p_1 + p_2 - p_1p_2}$.\\
So above all, the expectation time until at least one has a success is $E(A) = \dfrac{1}{p_1 + p_2 - p_1p_2}$.\\


(c) Let $p = p_1 = p_2$.\\
Let B: the first time that Mario is successful.\\
Let C: the first time that Zelda is successful.\\
D: their first sunccesses are simultaneous.\\
So $B\sim FS(p), Y\sim FS(p)$. And since B and C are independent, and $P(B = k) = (1-p_1)^{k-1}p_1$, $P(C = k) = (1-p_2)^{k-1}p_2$.\\
So $P(D) = P(B=C) = \sum\limits_{k=1}^{\infty}P(B=k,C=k)$
$ = \sum\limits_{k=1}^{\infty}P(B=k)P(C=k)$
$ = \sum\limits_{k=1}^{\infty}(1-p_1)^{k-1}p_1(1-p_2)^{k-1}p_2$
$ = \sum\limits_{k=0}^{\infty}((1-p)^2)^kp^2 = p^2\cdot \dfrac{1}{1-(1-p)^2} = \dfrac{p^2}{p(2-p)} = \dfrac{p}{2-p}$.\\
So the probability that their first time success are simultaneous is $\dfrac{p}{2-p}$.\\

And there are total three situations,\\
1. Mario get the first success earlier than Zelda, i.e. B<C\\
2. Zelda get the first success earlier than Mario, i.e. C<B\\
3. Mario and Zelda get the first success simultaneously i.e. B=C and it is exactly D that we mentioned above.\\
Since there are total three situations, so $P(B<C) + P(C<B) + P(B=C) = 1$.\\
Since $p_1 = p_2 = p$, so with symmetry we can get that $P(B<C) = P(C<B)$.\\
So $2P(B<C) + P(D) = 1$, i.e. $P(B<C)=\dfrac{1}{2}(1-P(D)) = \dfrac{1}{2}(1-\dfrac{p}{2-p}) = \dfrac{1}{2}\dfrac{2-2p}{2-p} = \dfrac{1-p}{2-p}$.\\
So the probability that Marios's first success orecedes Zelda's is $\dfrac{1-p}{2-p}$.\\\\

So above all, the probability that their first time success are simultaneous is $\dfrac{p}{2-p}$.\\
The probability that Marios's first success orecedes Zelda's is $\dfrac{1-p}{2-p}$.\\

\end{homeworkProblem}

\newpage

\begin{homeworkProblem}[4]

$X\sim Geom(p), Y\sim Geom(q)$.\\
So $P(X=k) = (1-p)^kp, P(Y=k)=(1-q)^kq$.\\

(a) $P(X=Y)=\sum\limits_{k=0}^{\infty}P(X=Y=k)=\sum\limits_{k=0}^{\infty}P(X=k)P(Y=k)$\\
$=\sum\limits_{k=0}^{\infty}(1-p)^kp(1-q)^kq=pq\sum\limits_{k=0}^{\infty}[(1-p)(1-q)]^k$\\
since $p\in (0,1),q\in (0,1)$, so $(1-p)(1-q)\in (0,1)$.\\
So $P(X=Y) = pq\cdot \dfrac{1}{1-(1-p)(1-q)} = \dfrac{pq}{p+q-pq}$.\\
So above all, the probability that $X=Y$ is $P(X=Y)=\dfrac{pq}{p+q-pq}$.\\

(b) Let $F(z)$ be the CDF of $max(X,Y)$, so $F(z) = \sum P(max(X,Y)\leq z) = P(X\leq z, Y\leq z)$\\
Since X, Y are independent,\\
so $F(z) = P(X\leq z)P(Y\leq z) = (\sum\limits_{k=0}^z (1-p)^kp)(\sum\limits_{k=0}^z (1-q)^kq) = [1-(1-p)^{z+1}][1-(1-q)^{z+1}]$.\\
And the Survival Function $G(z) = 1 - F(z) = (1-p)^{z+1} + (1-q)^{z+1} - [(1-p)(1-q)]^{z+1}$.\\
So $E(max(X,Y)) = \sum\limits_{z=0}^{\infty}G(z) = \dfrac{1-p}{p} + \dfrac{1-q}{q} - \dfrac{(1-p)(1-q)}{p+q-pq}$.\\
So above all, $E(max(X,Y)) = \dfrac{1-p}{p} + \dfrac{1-q}{q} - \dfrac{(1-p)(1-q)}{p+q-pq}$.\\

(c) $P(min(X,Y) = k) = \sum\limits_{i=k}^{\infty}P(X=k,Y=i) + \sum\limits_{i=k+1}^{\infty}P(X=i,Y=k)$\\
Since X,Y are independent, so $P(min(X,Y) = k) = (1-p)^kp\sum\limits_{i=k}^{\infty}(1-q)^iq+(1-q)^kq\sum\limits_{i=k+1}^{\infty}(1-p)^ip$\\
$=(1-p)^kpq\dfrac{(1-q)^k}{q} + (1-q)^{k}qp\dfrac{(1-p)^{k+1}}{p} = (1-p)^k(1-q)^kp + (1-p)^{k+1}(1-q)^kq$\\
$=(1-p)^k(1-q)^k(p+q-pq)$.\\
So above all, $P(min(X,Y) = k) = (1-p)^k(1-q)^k(p+q-pq)$.\\

(d) With the defination of the conditional probability, we have $$P(X=k|X\leq Y) = \dfrac{P(X=k,X\leq Y)}{P(X\leq Y)} = \dfrac{P(X=k)P(Y\geq k)}{P(Y\geq X)}$$
With the LOPT, we have $P(Y\geq X) = \sum\limits_{i=0}^{\infty}P(Y\geq X|X=i)P(X=i) = \sum\limits_{i=0}^{\infty}P(Y\geq i)P(X=i)$\\
And $P(Y\geq k)=\sum\limits_{i=k}^{\infty}(1-q)^iq=q\cdot \dfrac{(1-q)^k}{1-(1-q)}=(1-q)^k$.\\
So $P(Y\geq X) = \sum\limits_{i=0}^{\infty}(1-q)^i(1-p)^ip=p\cdot \dfrac{1}{1-(1-p)(1-q)}$.\\
So $P(X=k|X\leq Y) = \dfrac{P(X=k)P(Y\geq k)}{P(Y\geq X)} = \dfrac{(1-p)^kp(1-q)^k}{p\cdot \dfrac{1}{1-(1-p)(1-q)}} = (1-p)^k(1-q)^k(p+q-pq)$.\\\\

So $E[X|X\leq Y] = \sum\limits_{k=0}^{\infty}k\cdot P(X=k|X\leq Y) = (p+q-pq)\sum\limits_{k=0}^{\infty}k[(1-p)(1-q)]^k$.\\\\

Let's construct $S = \sum\limits_{k=0}^{\infty}kq^k$\\
then $q\cdot S = \sum\limits_{k=1}^{\infty}(k-1)q^k$.\\
minus the two equation, we can get $(1-q)S=q+q^2+q^3+\cdots=\dfrac{q}{1-q}$.\\
So $S = \dfrac{q}{(1-q)^2}$.\\\\ 

With this, we can get that $\sum\limits_{k=0}^{\infty}k[(1-p)(1-q)]^k=\dfrac{(1-p)(1-q)}{[1-(1-p)(1-q)]^2}$.\\
So $E[X|X\leq Y] = (p+q-pq)\cdot \dfrac{(1-p)(1-q)}{(p+q-pq)^2} = \dfrac{(1-p)(1-q)}{p+q-pq}$.\\

So above all, $E[X|X\leq Y] = \dfrac{(1-p)(1-q)}{p+q-pq}$.\\

\end{homeworkProblem}

\newpage

\begin{homeworkProblem}[5]
(a)  Let $A_i$ be the rank that the $i$-th dishes we have ordered in the exploration phase.\\
Let $A$ be the sum of the ranks that all dishes we have eaten.\\
so $A = A_1 + A_2 + \cdots + A_k + (m-k)\cdot X$.
Since during the exploration phase, all food we will only ordered once, so there must exist a single $j$, s.t. $A_j = X$.\\
So $A = \sum\limits_{i=1,i\neq j}^{k} A_i + A_j + (m-k)\cdot X$
$ = \sum\limits_{i=1,i\neq j}^{k} A_i + (m-k+1)\cdot X$.\\
So for all $i\neq j$, we have $1\leq A_i < X$,
so $E(A_i) = \dfrac{1}{E[X]-1}\cdot (1+2+\cdots+(E[X]-1))=\dfrac{E[X]}{2}$.\\
So $E(A) = E(\sum\limits_{i=1,i\neq j}^{k} A_i + (m-k+1)\cdot X)=\sum\limits_{i=1,i\neq j}^{k} E(A_i) + (m-k+1)\cdot E[X]$\\
$=(k-1)\cdot \dfrac{E[X]}{2}+(m-k+1)E[X]=(\dfrac{k+1}{2}+m-k)E[X]$.\\

So above all, $E(A) = (\dfrac{k+1}{2}+m-k)E[X]$.\\

(b) Since we have eaten $k$ dishes at the exploration phase, so the smallest rank we will get is $k$.\\
So for $a$ that $1\leq a < k$, $P(X=a) = 0$.\\
As for a that $l\leq a\leq n$, since $a$ is the highest rank we get, and we do not ear two same dishes at the exploration phase.
So other $k-1$ dishes have the rank in $\{1,2,\cdots,a-1\}$, so the number of combination that the dishes with the highest rank $a$
is that ${a-1} \choose {k-1}$, and we can arrange them in different order, which has $k!$ ways. So the number of total number that the dishes with highest rank of $a$ is $k!{a-1\choose k-1}$.\\
As for the number of combinations of all possible choices, we just need to choose $k$ dishes from all $n$ kinds of dishes, which is $n\choose k$. And we can arrange them in different order, which has $k!$ ways. So the number of all possible choices is $k!{n\choose k}$\\
So the PMF of X is that $P(X=a) = \dfrac{\#\ dishes\ with\ highest\ rank\ of\ a}{\# all\ possible\ choices} = \dfrac{k!\left(\begin{array}{l}a-1\\k-1\end{array}\right)}{k!\left(\begin{array}{l}n\\k\end{array}\right)} = \dfrac{\left(\begin{array}{l}a-1\\k-1\end{array}\right)}{\left(\begin{array}{l}n\\k\end{array}\right)}$.\\

So above all, the PMF of X is that\\
$P(X=a)=\dfrac{\left(\begin{array}{l}a-1\\k-1\end{array}\right)}{\left(\begin{array}{l}n\\k\end{array}\right)}$, if $k\leq a\leq n$,\\
$P(X=a)=0$, otherwise.\\

(c) With the defination of expectation,\\
$$E[X]=\sum\limits_{a=k}^{n}a\cdot P(X=a) = \sum\limits_{a=k}^{n}a\cdot \dfrac{\left(\begin{array}{l}a-1\\k-1\end{array}\right)}{\left(\begin{array}{l}n\\k\end{array}\right)} = \dfrac{1}{\left(\begin{array}{l}n\\k\end{array}\right)}\cdot \sum\limits_{a=k}^{n}a{{a-1}\choose {k-1}}=\dfrac{k}{\left(\begin{array}{l}n\\k\end{array}\right)}\cdot \sum\limits_{a=k}^{n}{a\choose k}$$

We can get a lemma that $\sum\limits_{a=k}^{n}{a\choose k} = {n+1\choose k+1}$ with the story proof:\\
Consider a situation that there are a group of $n+1$ people, and we want to choose $k+1$ people from the group.\\ 
For the left-hand side, let each person has a unique number $1,2,\cdots,n+1$. So we can disguss about the biggset number of people in the subgroup.\\
If the biggest number in the subgroup is $a$, then for this situation, the number total ways to get the subgroup is to choose $n$ people from the number of $\{1,2,\cdots,a-1\}$, i.e. the number of ways is $a-1\choose k$.
Since the subgroup has total $k+1$ peole, so $a\geq k+1$. So the number of total ways is that $\sum\limits_{a=k+1}^{n}{a-1\choose k}=\sum\limits_{a=k}^{n}{a\choose k}$

For the right-hand side, we just directly choose the $k+1$ people from all $n+1$ people, so the number of total ways is $n+1\choose k+1$.\\
So $\sum\limits_{a=k}^{n}{a\choose k} = {n+1\choose k+1}$ has been proved.\\

Combine the lemma with the $E[X]$ that we have reduced above, we can get that
$$E[X] = \dfrac{k}{\left(\begin{array}{l}n\\k\end{array}\right)}\cdot {n+1\choose k+1}=\dfrac{k}{\dfrac{n!}{k!(n-k)!}}\dfrac{(n+1)!}{(k+1)!((n+1)-(k+1))!}=\dfrac{k(n+1)}{k+1}$$

So above all, $E[X] = \dfrac{k(n+1)}{k+1}$.\\

(d) Combine (a) and (c) we can get that $E(A)=(\dfrac{k+1}{2}+m-k)\dfrac{k(n+1)}{k+1}$\\
To get the optimal value of $k$, we just need to let the derivatives of $E(A)$ be $0$.\\
So let $f(k)=E(A)=(\dfrac{k+1}{2}+m-k)\dfrac{k(n+1)}{k+1}$, then $f'(k)=(n+1)(\dfrac{1}{2}-\dfrac{m+k^2+2k}{(k+1)^2})$.\\
Let $f'(k)=0$, since $n > 0$, so $n+1\neq 0$, so $\dfrac{1}{2}-\dfrac{m+k^2+2k}{(k+1)^2}=0$, simplify it we can get $(k+1)^2=2(m+1)$.
So $k=-1\pm\sqrt{2(m+1)}$.\\
Since $m\geq 1$, so $\sqrt{2(m+1)}\geq 2$, since $k>0$, so we can only take $k=-1+\sqrt{2(m+1)}$.\\
And take it into verify, $k=\sqrt{2(m+1)}-1$ is the optimal value.\\

So above all, the optimal value is $k=\sqrt{2(m+1)}-1$.\\

\end{homeworkProblem}

\newpage

\end{document}
