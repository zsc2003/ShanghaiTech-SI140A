\documentclass{article}

\usepackage{fancyhdr}
\usepackage{extramarks}
\usepackage{amsmath}
\usepackage{amsthm}
\usepackage{amsfonts}
\usepackage{tikz}
\usepackage[plain]{algorithm}
\usepackage{algpseudocode}
\usepackage{enumerate}
\usepackage{tikz}

\usetikzlibrary{automata,positioning}

%
% Basic Document Settings
%  

\topmargin=-0.45in
\evensidemargin=0in
\oddsidemargin=0in
\textwidth=6.5in
\textheight=9.0in
\headsep=0.25in

\linespread{1.1}

\pagestyle{fancy}
\lhead{\hmwkAuthorName}
\chead{\hmwkClass : \hmwkTitle}
\rhead{\firstxmark}
\lfoot{\lastxmark}
\cfoot{\thepage}

\renewcommand\headrulewidth{0.4pt}
\renewcommand\footrulewidth{0.4pt}

\setlength\parindent{0pt}

%
% Create Problem Sections
%

\newcommand{\enterProblemHeader}[1]{
    \nobreak\extramarks{}{Problem \arabic{#1} continued on next page\ldots}\nobreak{}
    \nobreak\extramarks{Problem \arabic{#1} (continued)}{Problem \arabic{#1} continued on next page\ldots}\nobreak{}
}

\newcommand{\exitProblemHeader}[1]{
    \nobreak\extramarks{Problem \arabic{#1} (continued)}{Problem \arabic{#1} continued on next page\ldots}\nobreak{}
    \stepcounter{#1}
    \nobreak\extramarks{Problem \arabic{#1}}{}\nobreak{}
}

\newcommand*\circled[1]{\tikz[baseline=(char.base)]{
		\node[shape=circle,draw,inner sep=2pt] (char) {#1};}}


\setcounter{secnumdepth}{0}
\newcounter{partCounter}
\newcounter{homeworkProblemCounter}
\setcounter{homeworkProblemCounter}{1}
\nobreak\extramarks{Problem \arabic{homeworkProblemCounter}}{}\nobreak{}

%
% Homework Problem Environment
%
% This environment takes an optional argument. When given, it will adjust the
% problem counter. This is useful for when the problems given for your
% assignment aren't sequential. See the last 3 problems of this template for an
% example.
%

\newenvironment{homeworkProblem}[1][-1]{
    \ifnum#1>0
        \setcounter{homeworkProblemCounter}{#1}
    \fi
    \section{Problem \arabic{homeworkProblemCounter}}
    \setcounter{partCounter}{1}
    \enterProblemHeader{homeworkProblemCounter}
}{
    \exitProblemHeader{homeworkProblemCounter}
}

%
% Homework Details
%   - Title
%   - Class
%   - Due date
%   - Name
%   - Student ID

\newcommand{\hmwkTitle}{Homework\ \#01}
\newcommand{\hmwkClass}{Probability \& Statistics for EECS}
\newcommand{\hmwkDueDate}{Feb 19, 2023}
\newcommand{\hmwkAuthorName}{Zhou Shouchen}
\newcommand{\hmwkAuthorID}{2021533042}


%
% Title Page
%

\title{
    \vspace{2in}
    \textmd{\textbf{\hmwkClass:\\  \hmwkTitle}}\\
    \normalsize\vspace{0.1in}\small{Due\ on\ \hmwkDueDate\ at 23:59}\\
	\vspace{4in}
}

\author{
	Name: \textbf{\hmwkAuthorName} \\
	Student ID: \hmwkAuthorID}
\date{}

\renewcommand{\part}[1]{\textbf{\large Part \Alph{partCounter}}\stepcounter{partCounter}\\}

%
% Various Helper Commands
%

% Useful for algorithms
\newcommand{\alg}[1]{\textsc{\bfseries \footnotesize #1}}
% For derivatives
\newcommand{\deriv}[1]{\dfrac{\mathrm{d}}{\mathrm{d}x} (#1)}
% For partial derivatives
\newcommand{\pderiv}[2]{\dfrac{\partial}{\partial #1} (#2)}
% Integral dx
\newcommand{\dx}{\mathrm{d}x}
% Alias for the Solution section header
\newcommand{\solution}{\textbf{\large Solution}}
% Probability commands: Expectation, Variance, Covariance, Bias
\newcommand{\E}{\mathrm{E}}
\newcommand{\Var}{\mathrm{Var}}
\newcommand{\Cov}{\mathrm{Cov}}
\newcommand{\Bias}{\mathrm{Bias}}

\begin{document}

\maketitle

\pagebreak

\begin{homeworkProblem}[1]

(a) consider a situation that we want to devide $n+1$ people into $k$ nonempty groups. 
We can use two different ways to count the number of different ways.\\

The left-hand side is directly get the number of ways that devide $(n+1)$ people into $k$
nonempty groups. i.e. the number is $\left\{\begin{array}{c}n+1\\k\end{array}\right\}$\\

And for the right-hand side, consider the $(n+1)-th$ person especially.\\
If one group only contains the $(n+1)-th$ person,
since there are total $k$ groups, and the $(n+1)-th$ person takes one group, so there have another $k-1$ groups remained,
so as for other $n$ people, there are $\left\{\begin{array}{c}n\\k-1\end{array}\right\}$ ways to devide the other $n$ people into $k-1$ nonempty groups.\\
Otherwise, the $n$ people are devided into $k$ nonempty groups. There are total $\left\{\begin{array}{c}n \\ k \end{array}\right\}$ ways,
and for the $(n+1)-th$ person, he can choose to join in any of the $k$ groups.\\
So in total, this situation has $k \cdot \left\{\begin{array}{c}n\\k \end{array}\right\}$ ways.\\

Totally, sum these two situations, we can get that the right-hand side has total $\left\{\begin{array}{c}n \\ k-1\end{array}\right\}+k\left\{\begin{array}{l}n \\ k\end{array}\right\}$ ways.\\

So above all, this situation can be solved in two ways, i.e. the left-hand side and the right-hand side.
So we have proved that $\left\{\begin{array}{c}n+1\\k\end{array}\right\}=\left\{\begin{array}{c}n \\ k-1\end{array}\right\}+k\left\{\begin{array}{l}n \\ k\end{array}\right\}$.\\

(b) consider a situation that we want to devide $n+1$ people into $k+1$ nonempty groups.
We can use two different ways to count the number of different ways. \\

For the left-hand side, consider the $(n+1)-th$ person especially. Let $j$ be the number of people that is not in the same group with the $(n+1)-th$ person.
Since all groups are nonempty, so $j \geq (k+1)-1$, i.e. $j \geq k$. Also, we have $j \leq (n+1)-1$, i.e. $j \leq n$. So $k\leq j\leq n$.
So what we need to do is to find the number that put $j$ people into $k$ nonempty groups. The first step, we need to pick $j$ people for the other $n$ people, which has
$n\choose j$ ways. And the second step, we need to put the $j$ people into $k$ nonempty groups, which has 
$\left\{\begin{array}{l}j\\k\end{array}\right\}$ ways. So combine them, for each number $j$, we have 
${n\choose j}\left\{\begin{array}{l}j\\k\end{array}\right\}$ ways.
And since $k\leq j\leq n$, so we have totally 
$\sum_{j=k}^n{n\choose j}\left\{\begin{array}{l}j\\k\end{array}\right\}$ ways.\\

And for the right-hand side is directly get the number that devide $n+1$ people into $k+1$
nonempty groups. i.e. the number is $\left\{\begin{array}{c}n+1\\k+1\end{array}\right\}$\\

So above all, this situation can be solved in two ways, i.e. the left-hand side and the right-hand side.
so we have proved that $\sum_{j=k}^n{n\choose j}\left\{\begin{array}{l}j \\ k\end{array}\right\}=\left\{\begin{array}{l}n+1\\k+1\end{array}\right\}$.\\

\end{homeworkProblem}

\newpage

\begin{homeworkProblem}[2]

The number of all norepeatwords that uses all $26$ letters is the arrangement of all 26 letters, i.e. $26!$\\
And for a norepeatwords that has $i$ letter(s), $1\leq i\leq 26$. Firstly, we should pick $i$ letter(s) for 
the all $26$ letters, which has $26\choose i$ ways.
And secondly, we can arrange all these $i$ letter(s), which has $i!$ ways.
So totally, for each $i$, the number of norepeatwords is ${26\choose i}i!$.
And since $1\leq i\leq 26$, so the number of all norepeatwords is $\sum_{i=1}^{26}{26\choose i}i!$\\

And since a norepeatword is chosen randomly, and all norepeatwords are equally likely,
so the probability is that:
\[
    P(norepeatword\ use\ all\ 26\ letters)
\]
\[
    =\dfrac{\#\ norepeatword\ use\ all\ 26\ letters}{\#\ all\ norepeatwords}
\]
\[
    =\dfrac{26!}{\sum_{i=1}^{26}{26\choose i}i!}
\]
\[
    =\dfrac{26!}{\sum_{i=1}^{26}\dfrac{26!}{i!(26-i)!}i!}
\]
\[
    =\dfrac{1}{\sum_{i=1}^{26}\dfrac{1}{(26-i)!}}
\]
\[
    =\dfrac{1}{\sum_{i=0}^{25}\dfrac{1}{i!}}
\]

From the knowledge of mathematical analysis, we knew that the Taylor expansion of $e^x$ is that\\
$e^x=\sum_{i=0}^{\infty} \dfrac{x^i}{i!}$\\
compute the value at $x=1$, and we can get that 
$e=\sum_{i=0}^{\infty} \dfrac{1}{i!}$\\
since when $i>25$, the loss of $\sum_{i=0}^{25} \dfrac{1}{i!}$ with $e$ is no bigger than $\dfrac{1}{25!}$, 
which is even smaller than $10^{-25}$\\
so we can say that $\sum_{i=0}^{25} \dfrac{1}{i!}$ is very close to $e$,\\
i.e. $\dfrac{1}{\sum_{i=0}^{25}\dfrac{1}{i!}}$ is very close to $\dfrac{1}{e}$,\\
i.e. the probability that norepeatword use all 26 letters is very close to $\dfrac{1}{e}$.\\

\end{homeworkProblem}

\newpage

\begin{homeworkProblem}[3]

(a)to get the number of valid curricula, firstly, we can choose $4$ lower level courses
from all $8$ lower level courses, which has $8\choose 4$ selections.\\
Secondly, we can choose $4$ higher level lessons from all $10$ lower level courses, which has $10\choose 3$ selections.\\
So the total number of possible curricula is \\
\[
    {8\choose 4}{10\choose 3}
\]
\[
    =  \dfrac{8!}{4!(8-4)!} \cdot \dfrac{10!}{3!(10-3)!}
\]
\[
    = 8400  
\]


(b) Since $L_1,L_2,L_3$ are prerequisites of some higher level courses, so we can have a discuss about whether $L_1,L_2,L_3$ are chosen.

\begin{enumerate}
    \item 
    $L_1$ is chosen, $L_2$ is chosen, $L_3$ is chosen\\
    so for lower level lessons, there are $5$ courses remained, and we need to choose $4-3=1$ from it,
    so the number of all selections is $5\choose 1$\\
    $\{H_1,\cdots,H_5\}$ can be chosen, $\{H_6,\cdots,H_{10}\}$ can be chosen, so we need to choose $3$ from
    $5+5=10$ courses, so the number of all selections is $\left(\begin{array}{l}10\\3\end{array}\right)$\\
    So the total number of selections of this case is that
    $\left(\begin{array}{l}5\\1\end{array}\right)\left(\begin{array}{l}10\\3\end{array}\right)=\dfrac{5!}{1!(5-1)!}\dfrac{10!}{3!(10-3)!}=600$

    \item 
    $L_1$ is chosen, $L_2$ is chosen, $L_3$ is not chosen\\
    so for lower level lessons, there are $5$ courses remained, and we need to choose $4-2=2$ from it,
    so the number of all selections is $\left(\begin{array}{l}5\\2\end{array}\right)$\\
    $\{H_1,\cdots,H_5\}$ can be chosen, $\{H_6,\cdots,H_{10}\}$ can not be chosen, so we need to choose $3$ from
    $5+0=5$ courses, so the number of all selections is $\left(\begin{array}{l}5\\3\end{array}\right)$\\
    So the total number of selections of this case is that
    $\left(\begin{array}{l}5\\2\end{array}\right)\left(\begin{array}{l}5\\3\end{array}\right)=\dfrac{5!}{2!(5-2)!}\dfrac{5!}{3!(5-3)!}=100$

    \item 
    $L_1$ is chosen, $L_2$ is not chosen, $L_3$ is chosen\\
    so for lower level lessons, there are $5$ courses remained, and we need to choose $4-2=2$ from it,
    so the number of all selections is $\left(\begin{array}{l}5\\2\end{array}\right)$\\
    $\{H_1,\cdots,H_5\}$ can be chosen, $\{H_6,\cdots,H_{10}\}$ can not be chosen, so we need to choose $3$ from
    $5+0=5$ courses, so the number of all selections is $\left(\begin{array}{l}5\\3\end{array}\right)$\\
    So the total number of selections of this case is that
    $\left(\begin{array}{l}5\\2\end{array}\right)\left(\begin{array}{l}5\\3\end{array}\right)=\dfrac{5!}{2!(5-2)!}\dfrac{5!}{3!(5-3)!}=100$

    \item 
    $L_1$ is chosen, $L_2$ is not chosen, $L_3$ is not chosen\\
    so for lower level lessons, there are $5$ courses remained, and we need to choose $4-1=3$ from it,
    so the number of all selections is $\left(\begin{array}{l}5\\3\end{array}\right)$\\
    $\{H_1,\cdots,H_5\}$ can be chosen, $\{H_6,\cdots,H_{10}\}$ can not be chosen, so we need to choose $3$ from
    $5+0=5$ courses, so the number of all selections is $\left(\begin{array}{l}5\\3\end{array}\right)$\\
    So the total number of selections of this case is that
    $\left(\begin{array}{l}5\\3\end{array}\right)\left(\begin{array}{l}5\\3\end{array}\right)=\dfrac{5!}{3!(5-3)!}\dfrac{5!}{3!(5-3)!}=100$

    \item 
    $L_1$ is not chosen, $L_2$ is chosen, $L_3$ is chosen\\
    so for lower level lessons, there are $5$ courses remained, and we need to choose $4-2=2$ from it,
    so the number of all selections is $\left(\begin{array}{l}5\\2\end{array}\right)$\\
    $\{H_1,\cdots,H_5\}$ can not be chosen, $\{H_6,\cdots,H_{10}\}$ can be chosen, so we need to choose $3$ from
    $0+5=5$ courses, so the number of all selections is $\left(\begin{array}{l}5\\3\end{array}\right)$\\
    So the total number of selections of this case is that
    $\left(\begin{array}{l}5\\2\end{array}\right)\left(\begin{array}{l}5\\3\end{array}\right)=\dfrac{5!}{2!(5-2)!}\dfrac{5!}{3!(5-3)!}=100$

    \item 
    $L_1$ is not chosen, $L_2$ is not chosen, $L_3$ is chosen\\
    $\{H_1,\cdots,H_5\}$ can not be chosen, $\{H_6,\cdots,H_{10}\}$ can not be chosen.\\
    So no higher level courses can be chosen.\\
    So the total number of selections of this case is $0$

    \item 
    $L_1$ is not chosen, $L_2$ is chosen, $L_3$ is not chosen\\
    $\{H_1,\cdots,H_5\}$ can not be chosen, $\{H_6,\cdots,H_{10}\}$ can not be chosen.\\
    So no higher level courses can be chosen.\\
    So the total number of selections of this case is $0$

    \item 
    $L_1$ is not chosen, $L_2$ is not chosen, $L_3$ is not chosen\\
    $\{H_1,\cdots,H_5\}$ can not be chosen, $\{H_6,\cdots,H_{10}\}$ can not be chosen.\\
    So no higher level courses can be chosen.\\
    So the total number of selections of this case is $0$
\end{enumerate}

So above all, the total number of different curricula is \\
$600+100+100+100+100+0+0+0$\\
$=1000$

\end{homeworkProblem}

\newpage

\begin{homeworkProblem}[4]

(a)since $k\leq 2$ is the number of people.\\
let event $A$ : "at least one birthday match"\\
so $A^c$ : "no birthday match".\\
And for $k$ given unique birthdays, there are $k!$ ways to match a student and a day.
、、So we have $P(A^c) = k!e_k(p_1,p_2,\cdots,p_{365})$.\\
So $P(A)=1-P(A^c)=1-k!e_k(p_1,p_2,\cdots,p_{365})$.\\
So above all, $P(no\ birthday\ match)=1-k!e_k(p_1,p_2,\cdots,p_{365})$\\
And since $\mathbf{p}=(p_1,p_2,\cdots,p_{365})$, so we can also say that\\
$P(at\ least\ one\ birthday\ match)=1-k!e_k(\mathbf{p})$\\


(b) For intuitively, $\forall j,p_j=\dfrac{1}{365}$ means that for every day in a year, the probability of people being born on each day is equal.
And this case is hard to match birthdays.\\
For the most extramely case, we say that $\exists i, p_i = 1$, and $\forall j \neq i, p_j = 0$, then everyone must born in the day $i$, and for $k\geq 2$, this case the
birthday must match. i.e. the possibility is $1$\\

(c)verify that:
$e_k(x_1,\cdots,x_n) = x_1x_2e_{k-2}(x_3,\cdots,x_n) + (x_1+x_2)e_{k-1}(x_3,\cdots,x_n) + e_k(x_3,\cdots,x_n)$\\
for each term of the elementary symmetric polynomial, if the term do not include any one of the $\{x_1,x_2\}$,
then that term must be in $e_k(x_3,\cdots,x_n)$\\
so we can say that each term in $e_k(x_1,\cdots,x_n) - e_k(x_3,\cdots,x_n)$ must include at least one of the $\{x_1,x_2\}$.\\
so we can say that
\[
    e_k(x_1,\cdots,x_n) - e_k(x_3,\cdots,x_n)
\]\[
    =\ \ \ \ \sum_{3\leq j_1 < j_2 < \cdots < j_{k-1} \leq n}x_1x_{j_1}\cdots x_{j_{k-1}}  
\]\[
    +\ \ \ \ \sum_{3\leq j_1 < j_2 < \cdots < j_{k-1} \leq n}x_2x_{j_1}\cdots x_{j_{k-1}}  
\]\[
    +\ \ \ \ \sum_{3\leq j_1 < j_2 < \cdots < j_{k-2} \leq n}x_1x_2x_{j_1}\cdots x_{j_{k-2}}  
\]
and since that\\
$e_{k-1}(x_3,\cdots,x_n) = \sum_{3\leq j_1 < j_2 < \cdots < j_{k-1} \leq n}x_{j_1}\cdots x_{j_{k-1}}$\\
$e_{k-2}(x_3,\cdots,x_n) = \sum_{3\leq j_1 < j_2 < \cdots < j_{k-2} \leq n}x_{j_1}\cdots x_{j_{k-2}}$\\
so \[
    e_k(x_1,\cdots,x_n) - e_k(x_3,\cdots,x_n)
\]\[
    = x_1e_{k-1}(x_3,\cdots,x_n) + x_2e_{k-1}(x_3,\cdots,x_n) + x_1x_2e_{k-2} (x_3,\cdots,x_n)
\]

so above all,  $e_k(x_1,\cdots,x_n) = x_1x_2e_{k-2}(x_3,\cdots,x_n) + (x_1+x_2)e_{k-1}(x_3,\cdots,x_n) + e_k(x_3,\cdots,x_n)$ is proved.\\
\\\\

After that, we want to show that $P(at\ least\ one\ birthday\ match\ |\ \mathbf{p}) \geq P(at\ least\ one\ birthday\ match\ |\ \mathbf{r})$.\\
since $r_1=r_2=\dfrac{p_1+p_2}{2}$, and for $3\leq j\leq 365$, $r_j = p_j$,\\
so $$e_k(p_3,\cdots,p_{365}) = e_k(r_3,\cdots,r_{365})$$
$$e_{k-1}(p_3,\cdots,p_{365}) = e_{k-1}(r_3,\cdots,r_{365})$$
$$e_{k-2}(p_3,\cdots,p_{365}) = e_{k-2}(r_3,\cdots,r_{365})$$
$$ p_1 + p_2 = r_1 + r_2 $$

So $$P(at\ least\ one\ birthday\ match\ |\ \mathbf{p}) - P(at\ least\ one\ birthday\ match\ |\ \mathbf{r})$$
$$ = (1-k!e_k(\mathbf{p})) - (1-k!e_k(\mathbf{r})) $$
$$ = k!(e_k(\mathbf{r}) - e_k(\mathbf{p})) $$
$$ = k!((r_1r_2e_{k-2}(r_3,\cdots,r_{365}) + (r_1+r_2)e_{k-1}(r_3,\cdots,r_{365}) + e_{k-1}(r_3,\cdots,r_{365}))$$
$$ - (p_1p_2e_{k-2}(p_3,\cdots,p_{365}) + (p_1+p_2)e_{k-1}(p_3,\cdots,p_{365}) + e_{k-1}(p_3,\cdots,p_{365}))) $$
$$ = k!e_{k-2}(p_3,\cdots,p_{365})(r_1r_2 - p_1p_2)$$
$$ = k!e_{k-2}(p_3,\cdots,p_{365})((\dfrac{p_1+p_2}{2})^2 - p_1p_2)$$

And since $k! > 0, \forall j, p_j,r_j \geq 0$, so $e_{k-2}(p_3,\cdots,p_{365})\geq 0$\\
also, we have $\dfrac{x+y}{2}\geq \sqrt{xy}$\\
so $(\dfrac{p_1+p_2}{2})^2 \geq p_1p_2$\\
so above all,
$$ k!e_{k-2}(p_3,\cdots,p_{365})((\dfrac{p_1+p_2}{2})^2 - p_1p_2) \geq 0$$
$$i.e. P(at\ least\ one\ birthday\ match\ |\ \mathbf{p}) \geq P(at\ least\ one\ birthday\ match\ |\ \mathbf{r})$$ 
is proved.\\

When $e_{k-2}(p_3,\cdots,p_{365}) = 0$, $P(at\ least\ one\ birthday\ match\ |\ \mathbf{p}) = P(at\ least\ one\ birthday\ match\ |\ \mathbf{r})$\\
But $e_{k-2}(p_3,\cdots,p_{365}) = 0$ is a really tough condition which is almost impossible to achieve.\\

And when $e_{k-2}(p_3,\cdots,p_{365}) \neq 0$,\\
If and only if when $p_1 = p_2, i.e. \mathbf{p} = \mathbf{r}$, 
$P(at\ least\ one\ birthday\ match\ |\ \mathbf{p}) = P(at\ least\ one\ birthday\ match\ |\ \mathbf{r})$\\\\

At last, we need to prove that the value of $\mathbf{p}$ that minimizes the probability
of at least one birthday match is given by $\forall j,p_j = \dfrac{1}{365}$\\

Suppose that $\mathbf{p}$ minimizes the probability of at least one birthday match.\\
If $\mathbf{p}$ exist two days have different possibilities, i.e. $\exists i,j\in [1,365]\cap \mathbb{Z}, i\neq j, p_i \neq p_j$,\\
and we can modify their index to make that regard $p_i$ as $p_1$ and $p_j$ as $p_2$.\\
And from the provement above, we can construct a $\mathbf{r}$ s.t.  $r_1=r_2=\dfrac{p_1+p_2}{2}$, and for $3\leq j\leq 365$, $r_j = p_j$.\\
Since $p_1 \neq p_2$, so the equally condition of the inequality is not satisfied.\\
So $P(at\ least\ one\ birthday\ match\ |\ \mathbf{p}) > P(at\ least\ one\ birthday\ match\ |\ \mathbf{r})$\\
With this construction, we should find that $\mathbf{p}$ does not minimize the probability of at least one birthday match
because of the existence of $\mathbf{r}$. Which is contradiction.\\
So the condition that $\mathbf{p}$ exist two days have different possibilities is not satisfied.\\
So $\forall i,j\in [1,365]\cap \mathbb{Z}, i\neq j, p_i = p_j$\\
So above all,  $\forall i\in [1,365]\cap \mathbb{Z}, p_i = \dfrac{1}{365}$.\\

\end{homeworkProblem}

\newpage

\begin{homeworkProblem}[5]

(a)Suppose that we are facing a situation that there are total $n+1$ steps,
and we can randomly choose any of the two different steps from all $n+1$ steps. And goes from the lower one to the higher one.
All steps are equally likely, and we want to count how many different ways that we can go.\\

For the left-hand side, we are considering the lower steps. Since the $(n+1)-th$ step cannot be the lower one, 
So for the $i-th$ step, $1\leq i\leq n$.\\
Since we consider the $i-th$ as the lower one, so the higher one could be one of the $\{(i+1),\cdots,n,(n+1)\}$-th steps.
And there are $(n+1)-(i+1)+1=n-i+1$ different selections.\\
And since $1\leq i\leq n$, so the number of total selections is that $\sum_{i=1}^{n}n-i+1=\sum_{i=1}^{n}i=1+2+\cdots+n$.\\

And for the right-hand side, we can just choose two steps from all $n+1$ steps, and let the smaller number be the lower step,
the bigger number be the higher step.\\
So the number of the selections is $n+1\choose 2$\\

So above all, this situation can be solved in two ways, i.e. the left-hand side and the right-hand side.\\
So we have proved that $1+2+\cdots+n={n+1\choose 2}$\\

(b)Suppose that we are facing a situation that there are total $n+1$ balls, and their index are $1,2,\cdots,n+,n+1$.\\
And what we need to do is to choose a number from $2$ to $n+1$, and named that number as $i$.\\
Then, we pick 3 balls with replacement, with a strict condition that all index of the balls that we picked up must be less than $i$.\\
Which means that if the balls' index are $j,k,l$, then $1 \leq j, k, l < i$, and $j,k,l$ could be the same.\\

For the left-hand side, for a given $i$, the three balls have $(i-1)^3$ choices because of the repalcement.
And since $2 \leq i\leq n+1$, so the number of total choices is
$$\sum_{2}^{n+1} (i-1)^3 = \sum_{1}^{n} i^3$$
$$= 1^3 + 2^3 + \cdots + n^3$$\\
\\

And for the right-hand side, we can choose 4 numbers from $1$ to $n+1$ at one time.(The numbers could be the same)\\
And consider the largest number as $i$. Then there will be four different situations.
\begin{itemize}
    \item there is only one number in the four chosen numbers.\\
        since for $j,k,l$, they must be lower than $i$, so this situation will not hold for the background we have already set.

    \item there are two distinct numbers in the four chosen numbers.\\
    So there are ${n+1\choose 2}$ ways to choose the two different numbers.\\
    Since $j , k ,l < i$, so it must has $j=k=l$, and $i$ be the bigger number.\\
    so the number of total choices of this situation is ${n+1\choose 2}$ 

    \item there are three distinct numbers in the four chosen numbers.\\
    so there are ${n+1\choose 3}$ ways to choose the two different numbers.\\
    Since $j,k,l$ have order, so we need to arrange them to give their the values.\\
    
    Since there are two distinct numbers in $j,k,l$, so first we need to pick out that which two index have the same number,
    and the number of the choice is ${3\choose 2}$\\
    After that, the two different numbers could have an arrangement, which has $2!$ different choices.\\
    so there are ${3\choose 2}2! = \dfrac{3!}{2!(3-2)!}2! = 6$ different arrangements,
    so the number of total choices of this situation is $6{n+1\choose 3}$ 


    \item there are four distinct numbers in the four chosen numbers.\\
    so there are ${n+1\choose 4}$ ways to choose the two different numbers.\\
    since $j,k,l$ have order, so we need to arrange them to give their the values.\\
    And there are $3! = 6$ different arrangements,
    so the number of total choices of this situation is $6{n+1\choose 4}$ 

\end{itemize}

In total, we can sum up these choices, so for the right-hand side, the number of the total choices is 
$6{n+1\choose 4}+6{n+1\choose 3}+{n+1\choose 2}$\\

So above all, this situation can be solved in two ways, i.e. the left-hand side and the right-hand side.\\
So we have proved that, $1^3+2^3+\cdots+n^3=$
$6{n+1\choose 4}+6{n+1\choose 3}+{n+1\choose 2}$

\end{homeworkProblem}

\end{document}
